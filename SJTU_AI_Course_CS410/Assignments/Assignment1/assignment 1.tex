\documentclass{article} 
\usepackage[fleqn]{amsmath}
\usepackage{graphicx}
\usepackage{mathrsfs}
\usepackage{color}
\usepackage{amssymb}
\usepackage{amsfonts,amssymb}
\usepackage{algorithm}
\usepackage{algorithmic}
\usepackage{array}



\title{\normalsize
CS410: Artificial Intelligence 2020 Fall\\
Homework 1: Search Algorithms\\
Due date: 23:59:59 (GMT +08:00), October 3 2020}
\author{}
\date{}

\begin{document} 
\maketitle

\noindent Consider the following simplified version of the classic Atari video game, \textit{Montezuma's Revenge}: It is played on the board illustrated below. An agent (represented by the person icon in cell (1,3)) wishes to grab the key (in cell (3,0)). A skull starts in cell (5,2) and moves to the right by one cell after each action is executed until it ends up in the rightmost cell, at which point it starts moving to the left, and repeats this pattern back and forth.\\

\noindent The agent can be facing either left or right. There are 10 possible actions for the agent: 2 turning actions (\textit{turn left,turn right}) and 8 moving actions (\textit{left,right,up,down,left up,left down, right up,right down}). The agent can move up or down while facing either direction, but can move sideways or diagonally only if facing in that direction. For example, if the agent is facing right but tries to move left up, the agent will not move and nothing will happen. Furthermore, if the agent is already facing left and a turn left action is taken, nothing happens.\\

\noindent Lastly, the agent cannot move into a cell \textbf{currently occupied} by the skull, or a wall.\\

\begin{figure}[h]
\centering
\includegraphics[width=7cm,height=7cm]{maze1.png}
\end{figure}

% {\centering\includegraphics[width=6cm,height=6cm]{maze1.png}}

\newpage
\noindent 1. Answer the following questions for the Montezuma’s revenge board above: \\

\noindent (1.a) Let $N$ be the number of possible cell locations that the agent can be in, and let $M$ be the number of possible cell locations that the skull can be in. Recall that for ``pacman pathing'', the representation of the state was $(x, y)$ where $x$ was the row and $y$ was the column of pacman's position.
Describe a representation of a state in the state space for this game and give an expression for the size of the state space. \\
~\\
~\\
Representation of the state space: \\
~\\
~\\
~\\
~\\
~\\
~\\
~\\
~\\
Size of the state space:\\
~\\
~\\
~\\
~\\
~\\
~\\
~\\
~\\
Explanation of each term in the size of the state space:\\
~\\
~\\
~\\
~\\
~\\
~\\
~\\
~\\
\newpage

\noindent (1.b) Please fill in the following pseudocode for the \textbf{getSuccessor} function for this game:
\begin{algorithm}[thb!]
\caption{getSuccessor(state)}
\begin{algorithmic}[1]
\STATE \textit{successors} $\gets$ empty \textit{list}
\FOR {action $\in \_ \_ \_ \_\_ \_ \_ \_\_ \_ \_ \_\_ \_ \_ \_\_ \_ \_ \_\_ \_ \_ \_\_ \_ \_ \_\_ \_ \_ \_\_ \_ \_ \_\_ \_ \_ \_\_ \_ \_ \_$}
	\IF{action $\in$ \{\textit{left,left\_up,left\_down}\} and \textit{state.facing\_direction} $==\_ \_\_ \_ \_ \_\_ \_ \_ \_$}
		\STATE continue
	\ENDIF
	\IF{action $\in$ \{\textit{right,right\_up,right\_down}\} and \textit{state.facing\_direction} $==\_ \_\_ \_ \_ \_\_ \_ \_ \_$}
		\STATE continue
	\ENDIF
	\IF{action $\in$ \{\textit{turn\_right}\} and \textit{state.facing\_direction} $==\_ \_\_ \_ \_ \_\_ \_ \_ \_$}
		\STATE continue
	\ENDIF
	\IF{action $\in$ \{\textit{turn\_left}\} and \textit{state.facing\_direction} $==\_ \_\_ \_ \_ \_\_ \_ \_ \_$}
		\STATE continue
	\ENDIF
	\STATE \textit{next\_state} $\gets$ \textit{state.apply\_action(action)}
	\IF{\textit{next\_state} not out of bound and \textit{next\_state}\_\_\_\_\_\_\_\_\_\_\_\_\_\_\_\_ and \textit{next\_state}\_\_\_\_\_\_\_\_\_\_\_\_\_\_\_\_ }
		\STATE \textit{successors}.append(\textit{next\_state})
	\ENDIF
\ENDFOR
\RETURN \textit{successors}
\end{algorithmic}
\end{algorithm}
~\\
~\\

\noindent (1.c) What is the goal test?


\newpage
\noindent 2. Now, consider a simplified version of the board below, which has \textbf{no skull} and \textbf{no facing-direction for the agent} (i.e., the agent can move in any of the $8$ directions as long as it remains in the board). For the four following graph search algorithms, perform the search procedure yourself and provide answers to the questions below regarding the nodes expanded during the search as well as the final path found by the algorithm.\\

\noindent On this board, assume that a diagonal move has a cost of $3$, whereas moving left, right, up, or down has a cost of $1$. Do notice the difference in costs, and recall which algorithms use this cost information and which algorithms do not.\\

\noindent Remember that the search procedure should begin at the agent’s starting position (C). To break ties when adding nodes of equal cost to the fringe, follow alphabetical order. \\

\noindent Finally, when listing the order/number of nodes expanded, do not include nodes which are taken off the fringe but discarded immediately due to already having been visited.


\begin{figure}[h]
\centering
\includegraphics[width=4cm,height=8cm]{maze2.png}
\end{figure}

\newpage

\noindent (2.a) \textbf{Breadth first graph search}
Recall that BFS computes the smallest number of steps, $b(v)$, taken to get to a node $v$ from the start node.\\
~\\
Order of nodes expanded:\\
~\\
~\\
~\\
~\\
~\\
Number of nodes expanded:\\
~\\
~\\
~\\
~\\
~\\
Path returned:\\
~\\
~\\
~\\
~\\
~\\
Length of path:\\
~\\
~\\
~\\
~\\
~\\
Cost of path:\\
~\\
~\\
~\\
~\\
~\\
Please fill in the form below.
\renewcommand\arraystretch{1.5}
\begin{table}[tbh!]
\begin{center}
    \begin{tabular}{|p{2cm}| p{1cm}| p{1cm} | p{1cm}| p{1cm} | p{1cm}| p{1cm} | p{1cm}| p{1cm}|}
\hline
    State $s$  & A & B & C & D & E & F & G & H \\ \hline
    $b(s)$ &   &   &   &  &   &   &   &  \\\hline
    length from $C$ to $s$  &   &   &   &  &   &   &   &  \\\hline
    cost from $C$ to $s$ &   &   &   &  &   &   &   &  \\
\hline
\end{tabular}
\end{center}
\end{table}

\newpage


\noindent (2.b) \textbf{Depth first graph search}\\
~\\
Order of nodes expanded:\\
~\\
~\\
~\\
~\\
~\\
Number of nodes expanded:\\
~\\
~\\
~\\
~\\
~\\
Path returned:\\
~\\
~\\
~\\
~\\
~\\
Length of path:\\
~\\
~\\
~\\
~\\
~\\
Cost of path:\\
~\\
~\\
~\\
~\\
~\\
Please fill in the form below.
\renewcommand\arraystretch{1.5}
\begin{table}[tbh!]
\begin{center}
    \begin{tabular}{|p{2cm}| p{1cm}| p{1cm} | p{1cm}| p{1cm} | p{1cm}| p{1cm} | p{1cm}| p{1cm}|}
\hline
    State $s$  & A & B & C & D & E & F & G & H \\ \hline
    length from $C$ to $s$  &   &   &   &  &   &   &   &  \\\hline
    cost from $C$ to $s$ &   &   &   &  &   &   &   &  \\
\hline
\end{tabular}
\end{center}
\end{table}



\newpage


\noindent (2.c) \textbf{Uniform cost graph search}\\
Recall that UCS keeps track of the lowest cost, $c(v)$, to get from the start node to the node $v$.
~\\ 
Order of nodes expanded:\\
~\\
~\\
~\\
~\\
~\\
Number of nodes expanded:\\
~\\
~\\
~\\
~\\
~\\
Path returned:\\
~\\
~\\
~\\
~\\
~\\
Length of path:\\
~\\
~\\
~\\
~\\
~\\
Cost of path:\\
~\\
~\\
~\\
~\\
~\\
What is $c(A), c(B), \ldots c(H)$?
\renewcommand\arraystretch{1.5}
\begin{table}[tbh!]
\begin{center}
    \begin{tabular}{|p{2cm}| p{1cm}| p{1cm} | p{1cm}| p{1cm} | p{1cm}| p{1cm} | p{1cm}| p{1cm}|}
\hline
    State $s$  & A & B & C & D & E & F & G & H \\ \hline
    $c(s)$ &   &   &   &  &   &   &   &  \\
\hline
\end{tabular}
\end{center}
\end{table}



\newpage
\noindent (2.d) \textbf{A* graph search (with Manhattan distance to the goal as the heuristic)} Recall that A* computes $f(v)$ for the nodes $v$ that it expands, with $f (v) = c(v) + h(v)$ where $c(v)$ is the lowest cost to reach $v$ from the start node and $h(v)$ is an estimate of the distance from $v$ to the goal.
~\\ 
Order of nodes expanded during the search:\\
~\\
~\\
~\\
~\\
~\\
Number of nodes expanded during the search:\\
~\\
~\\
~\\
~\\
~\\
Path returned by the search:\\
~\\
~\\
~\\
~\\
~\\
Length of path returned by the search:\\
~\\
~\\
~\\
~\\
~\\
Cost of path returned by the search:\\
~\\
~\\
~\\
~\\
~\\
What is $f(A), f(B), \ldots f(H)$? Note that here, we are asking for the true $f(v)$ values as dictated by the definition, which is the value populated by the search algorithm only if it were to expand every node. This particular search problem doesn't end up expanding all nodes, so the $f(v)$ estimate maintained by the algorithm on the queue is not the true $f(v)$ value that we’re asking for. Hint: you can fill out these values directly by looking at the board.

\renewcommand\arraystretch{1.5}
\begin{table}[tbh!]
\begin{center}
    \begin{tabular}{|p{2cm}| p{1cm}| p{1cm} | p{1cm}| p{1cm} | p{1cm}| p{1cm} | p{1cm}| p{1cm}|}
\hline
    State $s$  & A & B & C & D & E & F & G & H \\ \hline
    $f(s)$ &   &   &   &  &   &   &   &  \\
\hline
\end{tabular}
\end{center}
\end{table}


\newpage
\noindent 3. Given your answers above, what are the qualitative differences between the results achieved by BFS, DFS, UCS, and A*? Which one finds the shortest path (in number of steps)? Which one finds the optimal path (in cost)?



\newpage
\noindent 4. For the same board and setting as part 2, give an example for each of the following types of heuristics. Please briefly explain why the example you chose satisfies the requested properties. \\
~\\
(4.a) Admissible and consistent.
Note: You can use a heuristic that we have frequently used in this class, or you can just assign any numbers that qualify as an admissible and consistent heuristic.
\renewcommand\arraystretch{1.5}
\begin{table}[tbh!]
\begin{center}
    \begin{tabular}{|p{2cm}| p{1cm}| p{1cm} | p{1cm}| p{1cm} | p{1cm}| p{1cm} | p{1cm}| p{1cm}|}
\hline
    State $s$  & A & B & C & D & E & F & G & H \\ \hline
    Heuristic $h(s)$ &   &   &   &  &   &   &   &  \\
\hline
\end{tabular}
\end{center}
\end{table}
~\\
Explanation: \\
~\\
~\\
~\\
~\\
(4.b) Admissible but inconsistent.
\renewcommand\arraystretch{1.5}
\begin{table}[tbh!]
\begin{center}
    \begin{tabular}{|p{2cm}| p{1cm}| p{1cm} | p{1cm}| p{1cm} | p{1cm}| p{1cm} | p{1cm}| p{1cm}|}
\hline
    State $s$  & A & B & C & D & E & F & G & H \\ \hline
    Heuristic $h(s)$ &   &   &   &  &   &   &   &  \\
\hline
\end{tabular}
\end{center}
\end{table}
~\\
Explanation: 
~\\
~\\
~\\
~\\
(4.c) Inadmissible and inconsistent.
\renewcommand\arraystretch{1.5}
\begin{table}[tbh!]
\begin{center}
    \begin{tabular}{|p{2cm}| p{1cm}| p{1cm} | p{1cm}| p{1cm} | p{1cm}| p{1cm} | p{1cm}| p{1cm}|}
\hline
    State $s$  & A & B & C & D & E & F & G & H \\ \hline
    Heuristic $h(s)$ &   &   &   &  &   &   &   &  \\
\hline
\end{tabular}
\end{center}
\end{table}
~\\
Explanation: 
~\\
~\\
~\\
~\\


\newpage
\noindent 5. In the previous questions, perhaps you used “relaxed” heuristics; that is, you estimated the true cost of something by evaluating the cost for a simpler version of the problem (which is easier to compute). For example, using euclidean distance would be a ``relaxed'' heuristic to estimate the length of the shortest path from Arad to Bucharest in Romania.\\
~\\
Formally, we will define a \textbf{relaxed heuristic} as a function on a state that returns a value that is always admissible and consistent. In this problem, we will consider two changes (``skull'' and ``teleportation'') to the board/game above, and we will reason about the effect of these changes on the consistency of heuristics.\\

\noindent (5.a) For this new version of the game, your friend Nancy suggests taking the old game setting from part 2. and now adding the ability for the agent to perform a maximum of $1$ ``teleportation'' action during the game. That is, on one of the agent's moves, it can choose to jump from its current state to any other non-goal state on the board.\\
~\\
(5.a.i) How does this new teleportation ability change the state space of the game from part 2. , which was $(x, y)$? Does anything need to be added or removed?\\
~\\
~\\
~\\
(5.a.ii) Nancy argues that in this new game, at least one previously consistent heuristic can become inconsistent. Is Nancy right? 
\begin{itemize}
\item[$\bigcirc$] Yes, and I will give an example below.
\item[$\bigcirc$] No, and I will provide a proof below. 
\end{itemize}
~\\
\textbf{Note}: we define heuristics for this problem as being a function of \textbf{only} the cell location: They cannot incorporate anything that did not exist in the old version of the game that we are comparing to.\\
~\\
If you believe Nancy is right, give an example of a heuristic that used to be consistent in the old game but is no longer consistent in this new game. Make sure to explain why it is no longer consistent (perhaps with a drawing of a board state and an explanation).
\renewcommand\arraystretch{1.5}
\begin{table}[tbh!]
\begin{center}
    \begin{tabular}{|p{2cm}| p{1cm}| p{1cm} | p{1cm}| p{1cm} | p{1cm}| p{1cm} | p{1cm}| p{1cm}|}
\hline
    State $s$  & A & B & C & D & E & F & G & H \\ \hline
    $h(s)$ &   &   &   &  &   &   &   &  \\
\hline
\end{tabular}
\end{center}
\end{table}
~\\
~\\
If you believe Nancy is wrong, provide an argument for why a heuristic that was consistent in the old game must also remain consistent in this new game. Be specific about your reasoning and use mathematical quantities such as heuristic costs of states $h(a)$ and true costs of transitions $c(ab)$.\\


\newpage
\noindent(5.b)  For this new version of the board, your friend Ethan suggests adding the skull back to the old board setting from part 2. , and having the skull move back and forth between the cells E and F.\\
~\\
(5.b.i) How does the presence of this skull change the state space of the game from part 2. , which was $(x, y$)? Does anything need to be added or removed?\\
~\\
~\\
~\\
(5.b.ii) Ethan argues that in this new board, at least one previously consistent heuristic can become inconsistent. Is Ethan right?
\begin{itemize}
\item[$\bigcirc$] Yes, and I will give an example below.
\item[$\bigcirc$] No, and I will provide a proof below. 
\end{itemize}
~\\
\textbf{Note}: we define heuristics for this problem as being a function of \textbf{only} the cell location: They cannot incorporate the location of the skull, since that did not exist in the old version of the board that we are comparing to.\\
~\\
If you believe Ethan is right, give an example of a heuristic that used to be consistent on the old board but is no longer consistent on this new board. Make sure to explain why it is no longer consistent (perhaps with a drawing of a board state and an explanation.
\renewcommand\arraystretch{1.5}
\begin{table}[tbh!]
\begin{center}
    \begin{tabular}{|p{2cm}| p{1cm}| p{1cm} | p{1cm}| p{1cm} | p{1cm}| p{1cm} | p{1cm}| p{1cm}|}
\hline
    State $s$  & A & B & C & D & E & F & G & H \\ \hline
    $h(s)$ &   &   &   &  &   &   &   &  \\
\hline
\end{tabular}
\end{center}
\end{table}
~\\
~\\
If you believe Ethan is wrong, provide an argument for why a heuristic that was consistent in the old board must also remain consistent in this new board. Be specific about your reasoning and use mathematical quantities such as heuristic costs of states $h(a)$ and true costs of transitions $c(ab)$.











\end{document} 